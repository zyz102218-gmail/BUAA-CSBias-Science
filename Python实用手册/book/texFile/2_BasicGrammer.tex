\chapter{Python基本语法与程序控制结构}
在讨论基本语法前,先让我们来简单了解一下本课程中,我们所主要涉及到的程序的基本结构:输入-处理-输出结构,即IPO结构。
顾名思义,IPO结构主要有三部分构成,先获取输入,而后对输入的数据进行处理,得到结果,按照要求输出。看起来非常的简单,
不错,我们课程中所涉及到的绝大部分任务都可以用这样的控制结构来实现。

这样便引出了我们设计程序所需要了解的基本语法和控制结构。获取输入时,如何指定程序获取来自键盘的输入,亦或者是读取磁盘中
的文件,进阶来看,获取其他进程发来的通信也可以看作一种输入\footnote{但本课程不涉及多线程}。获取输入后,如何将数据按照怎样
的组织形式存入内存,而后又如何处理这些数据,这都涉及到基本语法中的基本数据类型(以及数据结构),基本运算和程序控制结构。最后,
如何获得我们想要的输出,怎样控制输出等。解决了以上问题,便实现了一个IPO程序框架下的程序实例。

因此,本章将首先介绍Python中的变量、输入与输出,然后介绍基本数据类型及其操作,之后介绍程序控制结构、
函数的声明与调用,最后介绍最基本的数据结构。在本章的末尾,将会有一段对Python中类的简要介绍,这部分
内容并不是课内所要求掌握的,留给有兴趣的同学尝试理解。
\section{变量、输入与输出}
Python是一门弱类型语言。