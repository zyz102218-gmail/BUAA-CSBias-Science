\chapter{Python基本语法与程序控制结构}
\lstset{style=Python}
在讨论基本语法前,先让我们来简单了解一下本课程中,我们所主要涉及到的程序的基本结构:输入-处理-输出结构,即IPO结构。
顾名思义,IPO结构主要有三部分构成,先获取输入,而后对输入的数据进行处理,得到结果,按照要求输出。看起来非常的简单,
不错,我们课程中所涉及到的绝大部分任务都可以用这样的控制结构来实现。

这样便引出了我们设计程序所需要了解的基本语法和控制结构。获取输入时,如何指定程序获取来自键盘的输入,亦或者是读取磁盘中
的文件,进阶来看,获取其他进程发来的通信也可以看作一种输入\footnote{但本课程不涉及多线程}。获取输入后,如何将数据按照怎样
的组织形式存入内存,而后又如何处理这些数据,这都涉及到基本语法中的基本数据类型(以及数据结构),基本运算和程序控制结构。最后,
如何获得我们想要的输出,怎样控制输出等。解决了以上问题,便实现了一个IPO程序框架下的程序实例。

因此,本章将首先介绍Python中的变量、输入与输出,然后介绍基本数据类型及其操作,之后介绍程序控制结构、
函数的声明与调用,最后介绍最基本的数据结构。在本章的末尾,将会有一段对Python中类的简要介绍,这部分
内容并不是课内所要求掌握的,留给有兴趣的同学尝试理解。
\section{变量、输入与输出}
\subsection{变量}
变量,以及常量\footnote{Python中并没有声明常量的关键字,
也就是没有办法让一个变量的值保持不变。但从规范的角度来说,当变量名全部为大写字母时,则认为这个
变量是常量,其值不应当被改变,但解释器在执行程序的过程中,并不会强制要求这一点},
是在编程中用于存储值以及对象的基本单元。因此,我们在这里首先介绍变量的概念。
Python是一门弱类型语言,具有自动内存管理的特点。这意味着,在Python中声明一个变量时,不必指定
其变量类型,通俗来说,不用指定一个变量里要存储的值是什么类型的。直接对一个变量赋值,如果这个变量
已经被声明了,那么其会改变变量的值,否则,解释器会为这个变量划分内存空间,创建这个变量,而后向
这个变量赋值。如下所示:

\begin{lstlisting}[language=Python]
a = 10
# 创建变量a,类型为整形(int),并赋值为10
b = 3.14
# 创建变量b,类型为浮点型(float),并赋值3.14
a = b
# 将b的值赋给a,并改变a的类型为b的类型
# 现在a是浮点型(float)
\end{lstlisting}
可以看到,在Python中,一个变量的类型在程序执行过程中是可变的。
因此,在实际编程中,要特别注意这一点。否则会给debug带来困扰。

下面让我们来了解一下哪些变量名是合法的。一个合法的变量名可以由数字、字母、下划线以及汉字混合组成,
但变量名的开头不能是数字。同时,Python对变量名中的英文大小写是敏感的,比如\texttt{Movie}
和\texttt{movie}不会被认为是同一个变量。下划线可以作为变量名的开头,但这通常被用于声明库内
变量来使用,一般书写程序时不必使用下划线作为开头。

\subsubsection{基本数据类型}
我们首先介绍Python中的整形(int),浮点型(float)和字符串(str)。这三种常用内建数据类型最为基本,
也与自然语言吻合的最好,因此首先介绍。

整形只能用于存放整数,其构造函数为\texttt{int}。当然,在Python中,一般情况下我们不需要使用构造
\subsubsection{变量命名规范}
既然讲到了变量的声明,那么我们不妨在这里简单聊一下关于变量名的代码规范。
对变量名的最基本要求很简单——见文知义,即自己或者他人看到这个变量名后,能够知道
这个变量的值所代表的含义,比如\texttt{age}或者\texttt{nianling}代表年龄\footnote{但尽量不要使用拼音},
\texttt{movieList}可以代表电影列表等等。尽量避免大量使用一个字母,或者一两个字母的简单变量名\footnote{当然,对于一些约定俗成的变量,
比如\texttt{i}、\texttt{j}、\texttt{k}作为迭代器迭代次数,\texttt{f}作为打开文
件的句柄这类名称是可以被接受的。}。
否则可能对后期代码维护,以及老师阅卷带来不必要的麻烦,甚至造成看不到得分点。

下面我们简单介绍一下Python中变量名的规范取法。通常,在Python中使用小驼峰命名法或蛇形命名法。下面我们来逐一介绍。

驼峰命名法是指,对于一个变量名,组成其含义的各个单词第一个首字母大写作为分割,比如:
\begin{lstlisting}[language=Python]
RawData # 原始数据
ArticleList # 文章列表
\end{lstlisting}
驼峰法有一种变体,被称为小驼峰,即第一个单词的首字母不大写,后面出现的单词的首字母大写,如:
\begin{lstlisting}[language=Python]
rawData = NULL
articleList = NULL
\end{lstlisting}

下面介绍蛇形命名法,蛇形命名法是将各个单词用下划线分割开来。在通常的蛇形命名法中,
不需要对单词的首字母做大小写处理。下面是两个示例:
\begin{lstlisting}[language=Python]
raw_data
article_list
\end{lstlisting}
而如果变量名以一个下划线起始,这通常代表这个变量是类的内部变量,不应该在类外访问,即这个
变量是被保护的\footnote{这并不意味着解释器会保护这个变量,
但在类外随意修改这类变量的值可能带来程序运行异常}。如下所示\footnote{摘自matplotlib源码}:
\begin{lstlisting}[language=Python]
_charsize_cache
_bbox_patch
\end{lstlisting}

也可以将驼峰法和蛇形命名法结合起来一同使用,在此不在赘述。

在本课程中,具体选用哪种变量的命名规范取决于你自己,甚至是否遵循PEP 8都并不是那么重要的。
遵守变里命名规范,以及代码规范最终的目的是让自己的代码保持一定的可读性,
一方面方便自己回看复习,另一方面方便我们进行实验指导以及阅卷。

