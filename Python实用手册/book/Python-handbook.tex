\documentclass[UTF8,a5paper,10pt]{ctexbook}

\usepackage[utf8]{inputenc}
% \usepackage{ctex} %导入中文包
\usepackage{geometry} %设置页边距的包
\usepackage{enumitem}
\usepackage{graphicx}
\usepackage{pdfpages} %本文最重要的一个包,就是将PDF文件加入到封面位置
\usepackage{ulem}
\makeatletter
\renewcommand\tableofcontents{%
    \thispagestyle{empty}
    \section*{\contentsname
        \@mkboth{%
           \MakeUppercase\contentsname}{\MakeUppercase\contentsname}}%
    \@starttoc{toc}%
    \newpage
    }

\geometry{left=2.5cm,right=2cm,top=2.54cm,bottom=2.54cm} %设置书籍的页边距

\title{大学计算机基础(理科)Python实用手册}
\author{Dill Chu编著}

\begin{document}
\section*{前言}
\addcontentsline{toc}{chapter}{前言}
\pagenumbering{roman}
\setcounter{page}{1}

大学计算机基础课程是计算机学院承担教学任务,
在北航学院托管开设的信息大类基础课程之一,是全体理科大类及强基计划,拔尖计划相关专业的学生在大一时应当修习的一门基础课程。
课程主要内容涵盖{计算机组成原理},{Python基础语法},{程序控制结构},{算法基础},{图表绘制},{数据分析},{GUI设计}等内容。
课程内容丰富,基本涵盖了理科大类对口各专业
在未来使用计算机作为工具,在实际科研或工作中解决问题的需求,或是提供了较好的入门指导工作,可以培养学习者的计算思维和运用计算机解决具体问题的能力。无论是否有计算机使用经验,是否有程序编写或脚本编写经历,在学习本课程后
都将具有相应的能力。

具体的课程内容与任务方面。课程主要包括2次讨论课,10次课内实验和1个大作业,所有题目都是助教同学利用寒假时间全新准备的。具体内容见下。

两次讨论课主要的内容是对图论
(七桥问题等的理解,最短路径问题等),最优化问题(枚举法,匈牙利算法等)等特定,相对较为复杂的算法与模型的理解和应用,
由每个人独立思考,而后在小组内讨论并向全班汇报解决思路与方案,课后每个人需要形成\dotuline{\textbf{独立的实验报告}}。

课内实验是跟随理论课程授课内容,对理论课上所讲授的知识进行上机运用的过程环节。课内共有10次实验,课内实验难度不是很大,最最重要的是\dotuline{\textbf{独立完成}}(包括提交到平台上的代码和实验报告)。

具体实验安排如下:
\begin{enumerate}[itemsep=2pt,topsep=0pt,parsep=0pt,start=0]
    \setlength{\itemsep}{-0.1cm}
    \item Python编程环境的使用
    \item Python基本语法
    \item 程序控制结构与列表
    \item 问题的描述—基本数据结构(字符串,字典)
    \item 问题的描述—自定义数据结构(栈,队列)
    \item 基本算法设计与实现(枚举,递归)
    \item 较复杂算法设计与实现(贪心法,动态规划)
    \item Python实现数据图形绘制
    \item 插值,拟合计算
    \item Pandas数据分析与GUI设计
\end{enumerate}
其中,前几次实验(实验0-实验2)的难点在于对Python最基础的语法的理解与运用,难在入门。这些内容并不是很难,甚至可以称得上非常简单。只是可能较为新鲜,此前没有接触过,才会带来一定的困扰。
只要实际上手写过一两个程序,这部分内容就可以很轻松地掌握。

在基础数据结构和算法部分(实验3-实验6),对知识本身的理解与运用则会成为相对困难的内容,但本课程的目的在于
培养运用与实用的思维,并不在于要求对这些算法有多么深刻的理解,也不会要求大家去leetcode等算法训练平台做大量的题来理解和巩固,当然,少量做一些题目,来提高和巩固自己也是欢迎的。
事实上,贪心法和动态规划都是计算机科学发展到现在,称得上是最基本,最简单而常用的计算机算法了,所以其难度并不是那么大,还请大家多多尝试学习和理解。

最后的几次实验(实验7-实验9),我认为这些是应用性较强的实验。直白来说,在这部分实验中,你并不需要知道多少原理,你要做的只是拿来几个模板,照猫画虎,按部就班地搭建“积木”,完成任务就可以了,并没有多少难度。
但鉴于Python是一门脚本语言,从笔者的实际经历来看,这些内容
在未来的实际科研和应用中,反而是非常非常有用的。

大作业通常在学期中后段布置,是一组具有不同难度系数,面向不同实用场景的实际问题的,较为系统和工程化的解决方案,是对整个学期课程所学的一次总体性的应用。在完成大作业的过程中,
往往需要跟随实验指导,自学一定量的知识。大作业题目具有一定的复杂性,所以也需要投入一定的时间去完成。通常在大作业后也会安排申优答辩环节,可以选择参加,需要花费一定的时间和精力去准备。

完成对课程内容和总体任务的介绍后,让我们回到编篡手册的目的上来。正如前面所说,本课程授课过程面临时间紧,任务重的困难,对于少部分没有计算机基础或使用经验的同学来说,学习这门课程的过程中可能会面临较大的困难。这往往体现在开展实验实操时进度不佳,或面对的困难较多而疲于应对。
而苦于授课教师和助教精力有限,分身乏术,在提供实验指导方面可能会有覆盖不到的地方,对部分同学的学习带来一定的压力,甚至是阻力。同时,实际线下实验指导时我们通常以口述方式进行指导,难以在讲述过程中配合相应的代码,最好的情况也是
配合同学当前所写的代码进行辅导。但这对描述问题和思路上时不利的。通常,配合代码示例的形式对问题和思路进行描述往往可以使读者(或者听众)对问题和思路形成更好的理解。
因此,我选择编篡本手册,一方面,是可以让同学们在每次实际进行实验前,通过学习实际题目和
代码的方式,能够能对这一部分的内容做大致了解,做到“手上有枪”;另一方面,我作为课程的修习者,以学生的视角编篡一份手册,
也是让大家能有一个相较于教材来讲,方便速查思路的材料。教材上的内容非常全面,高屋建瓴,但对于速查还是带来一定的困难,与同学们学习的思路上也不一定完全一致,而同龄人之间总是更容易找到相同的思路,产生一些共鸣。
\newpage
\tableofcontents
\thispagestyle{empty}
本书中所有代码,在Windows 11 23H2及macOS 14.2.1 23C71 arm64上以Anaconda 23.11.0, Python 3.12.0作为环境运行测试通过。具体第三方库版本见下:
\begin{enumerate}
    \item scipy
    \item numpy
    \item matplotlib
    \item pandas
\end{enumerate}
\newpage
\pagenumbering{arabic}
\chapter{Python编辑及运行环境配置}
\section{Anaconda的下载与安装}
\subsection{写在前面}
在本课程中,我们选择安装anaconda这一Python发行版进行学习与开发。一方面是因为anaconda在安装时不仅仅能够安装一个python解释器,
其也会同时安装ipython kernel,Jupyter notebook,matplotlib,scipy,numpy等多个常用的(也就是本课程中会用到的)第三方库。另一方面,anaconda提供Spyder这一高集成度,
便于初学者学习和调试的IDE,可以提供给我们“开箱即用”的体验\footnote{anaconda本身也提供很多别的功能,本章节后面会进行简要介绍}。

如果你想要挑战自己,当然可以自行到Python项目官网\footnote{https://python.org}上自行下载和安装Python解释器,而后使用pip包管理工具安装
我们所需的各类第三方库或者第三方组件(包括Spyder,也可以通过pip来自动安装)。但在正式开始教程之前,需要强调,作为初学者,请不要同时安装
纯净Python解释器和anaconda,即不要
让使用下载的Python安装包所安装的python解释器同anaconda在同一台电脑上并存,因为此时你的电脑里将同时有两个python运行环境,且这两个运
行环境没有通过环境管理器进行管理,其激活次序,即运行python脚本文件时正在使用的python解释器,或是尝试使用pip安装第三方库所指向的环境不易明确。
总之,作为课程的初学者,也是本手册后面的安排,我们将基于anaconda(conda+python解释器)进行介绍,包括环境配置,工具使用等,
所有代码也将会在这个环境下进行运行。
\subsection{Anaconda下载与安装}
Anaconda现已实现完善的公司化运营,可以访问其网站下载Anaconda:

https://www.anaconda.com/download

但通常,出于下载速度的考量,我们选择使用校园网联合镜像站的方式下载Anaconda,这不仅仅可以提供更快的安装包下载速度,同时其分发的
安装包也会将其中的默认镜像站替换成自己的分发站点,从而提供更高的pip与conda下载与更新速度。提供镜像站的组织有很多,如清华大学TUNA协会,中国科学技术大学所等高校及高校学生组织
所提供的下载源,此处以清华大学TUNA协会提供的下载源为例。

访问https://mirrors.tuna.tsinghua.edu.cn/anaconda/archive/ \footnote{此为双栈站点,可以自动选择IPv4或IPv6解析,将mirrors替换为mirrors6则只解析IPv6,替换为mirrors4则只解析IPv4,在校园网环境下,使用IPv6下载的文件不计流量。},
获取anaconda安装包列表。根据自己的操作系统版本和硬件架构,选择
下载对应的安装包体(如Windows-x86\_64对应AMD,
Intel处理器的Windows设备,MacOSX-arm64对应搭载Apple Silicon M1、M2、M3的设备)。
在当前的时间节点,建议选择Anaconda3-2023.09开头的安装包。如图\ref{fig:anacondaSelection}所示
\begin{figure}[htbp]
    \centering
    \includegraphics*[width=0.8\linewidth]{pic/tuna_anacondaDownload.png}
    \caption{Anaconda3安装包示例}
    \label{fig:anacondaSelection}
\end{figure}

对于不太熟悉命令行操作的同学来说,建议下载文件拓展名为pkg或exe的安装包,其可以提供图形化的安装界面和选项。具体安装过程在此不再赘述,
但请注意,\textbf{不要}把Anaconda安装在\dotuline{含有中文的目录}下,尤其注意你的Windows用户名是否带有中文。如果你不确定,可以直接将其安装在C盘外的其
他盘符\footnote{不要将Anaconda安装在C:/Program Files中},另外创建的文件夹中。

\subsection{环境变量与Spyder配置}

\subsubsection{环境变量配置}
配置环境变量的目的是令其他程序在运行python时能够直接找到base环境下的python,而不需要手动指定,同时也方便我们直接打开命令行就可以直接使用python解释器运行和调试程序。
对于windows和macOS而言,其配置环境变量的方式略有不同,下面分别介绍。

\textbf{Windows设备}

对于Windows设备而言,需要将anaconda安装目录下的bin/和Script

\subsubsection{Spyder}
安装流程结束后,在开始菜单(macOS:启动台)中寻找到Anaconda-Navigater,点击运行。
加载结束后,可以打开Spyder.
\end{document}
