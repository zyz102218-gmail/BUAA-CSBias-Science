\documentclass[UTF8,a5paper,10pt]{ctexbook}

\usepackage[utf8]{inputenc}
% \usepackage{ctex} %导入中文包
\usepackage{geometry} %设置页边距的包

\usepackage{pdfpages} %本文最重要的一个包,就是将PDF文件加入到封面位置
\usepackage{ulem}
\makeatletter
\renewcommand\tableofcontents{%
    \thispagestyle{empty}
    \section*{\contentsname
        \@mkboth{%
           \MakeUppercase\contentsname}{\MakeUppercase\contentsname}}%
    \@starttoc{toc}%
    \newpage
    }

\geometry{left=2.5cm,right=2cm,top=2.54cm,bottom=2.54cm} %设置书籍的页边距

\title{大学计算机基础(理科)Python实用手册}
\author{Dill Chu编著}

\begin{document}
\chapter*{第一版前言}
\addcontentsline{toc}{chapter}{前言}
\pagenumbering{roman}
\setcounter{page}{1}

大学计算机基础课程是计算机学院承担教学任务,
在北航学院托管开设的信息大类基础课程之一,是全体理科大类及相关学院强基计划、拔尖计划相关专业的学生在大一时应当修习的一门基础课程。
课程主要内容涵盖\dotuline{计算机组成原理}、\dotuline{编译原理}、\dotuline{Python基础语法}、\dotuline{程序控制结构}、\dotuline{算法基础}、\dotuline{图表绘制}、\dotuline{数据分析}、\dotuline{GUI设计}等内容。课程内容丰富,基本涵盖了理科大类对口各专业
在未来使用计算机作为工具,在实际科研或工作中解决问题的需求,或是提供了较好的入门指导工作,可以培养学习者的计算思维和运用计算机解决具体问题的能力。无论是否有计算机使用经验,是否有程序编写或脚本编写经历,在学习本课程后
都将具有相应的能力。

具体的课程内容与任务方面。课程主要

正如前面所说,本课程授课过程面临时间紧,任务重的困难,对于少部分没有计算机基础或使用经验的同学来说,学习这门课程的过程中可能会面临较大的困难。这往往体现在开展实验实操时进度不佳,或面对的困难较多而疲于应对。
而苦于授课教师和助教精力有限,分身乏术,在提供实验指导方面可能会有覆盖不到的地方,对部分同学的学习带来一定的压力,甚至是阻力。
\newpage
\tableofcontents

\end{document}
