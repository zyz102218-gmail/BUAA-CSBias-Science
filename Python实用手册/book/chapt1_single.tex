\documentclass[UTF8,a4paper,10pt]{ctexart}

\usepackage[utf8]{inputenc}
% \usepackage{ctex} %导入中文包
\usepackage{geometry} %设置页边距的包
\usepackage{enumitem}
\usepackage{graphicx}
\usepackage{pdfpages} %本文最重要的一个包,就是将PDF文件加入到封面位置
\usepackage{ulem}
\usepackage{listings}
\newcommand{\code}[1]{\texttt{#1}}
\makeatletter
\renewcommand\tableofcontents{%
    \thispagestyle{empty}
    \section*{\contentsname
        \@mkboth{%
           \MakeUppercase\contentsname}{\MakeUppercase\contentsname}}%
    \@starttoc{toc}%
    \newpage
    }
% 用来设置附录中代码的样式

\lstset{
    basicstyle          =   \sffamily,          % 基本代码风格
    keywordstyle        =   \bfseries,          % 关键字风格
    commentstyle        =   \rmfamily\itshape,  % 注释的风格,斜体
    stringstyle         =   \ttfamily,  % 字符串风格
    flexiblecolumns,                % 别问为什么,加上这个
    numbers             =   left,   % 行号的位置在左边
    showspaces          =   false,  % 是否显示空格,显示了有点乱,所以不现实了
    numberstyle         =   \zihao{-5}\ttfamily,    % 行号的样式,小五号,tt等宽字体
    showstringspaces    =   false,
    captionpos          =   t,      % 这段代码的名字所呈现的位置,t指的是top上面
    frame               =   lrtb,   % 显示边框
}

\lstdefinestyle{Python}{
    language        =   Python, % 语言选Python
    basicstyle      =   \zihao{-5}\ttfamily,
    numberstyle     =   \zihao{-5}\ttfamily,
    keywordstyle    =   \color{blue},
    keywordstyle    =   [2] \color{teal},
    stringstyle     =   \color{magenta},
    commentstyle    =   \color{black}\ttfamily,
    breaklines      =   true,   % 自动换行,建议不要写太长的行
    columns         =   fixed,  % 如果不加这一句,字间距就不固定,很丑,必须加
    basewidth       =   0.5em,
}

\lstdefinestyle{bash}{
    language        =   bash, % 语言选Python
    basicstyle      =   \zihao{-5}\ttfamily,
    numberstyle     =   \zihao{-5}\ttfamily,
    keywordstyle    =   \color{blue},
    keywordstyle    =   [2] \color{teal},
    stringstyle     =   \color{magenta},
    commentstyle    =   \color{black}\ttfamily,
    breaklines      =   true,   % 自动换行,建议不要写太长的行
    columns         =   fixed,  % 如果不加这一句,字间距就不固定,很丑,必须加
    basewidth       =   0.5em,
}
\geometry{left=2.5cm,right=2cm,top=2.54cm,bottom=2.54cm} %设置书籍的页边距

\title{大学计算机基础(理科)Python实用手册}
\author{Dill Chu编著}
\begin{document}
\chapter{Python编辑及运行环境配置}
\section{Anaconda的下载与安装}
\subsection{写在前面}
在本课程中,我们选择安装anaconda这一Python发行版进行学习与开发。一方面是因为anaconda在安装时不仅仅能够安装一个python解释器,
其也会同时安装ipython kernel,Jupyter notebook,matplotlib,scipy,numpy等多个常用的(也就是本课程中会用到的)第三方库。另一方面,anaconda提供Spyder这一高集成度,
便于初学者学习和调试的IDE,可以提供给我们“开箱即用”的体验\footnote{anaconda本身也提供很多别的功能,本章节后面会进行简要介绍}。

如果你想要挑战自己,当然可以自行到Python项目官网\footnote{https://python.org}上自行下载和安装Python解释器,而后使用pip包管理工具安装
我们所需的各类第三方库或者第三方组件(包括Spyder,也可以通过pip来自动安装)。但在正式开始教程之前,需要强调,作为初学者,请不要同时安装
纯净Python解释器和anaconda,即不要
让使用下载的Python安装包所安装的python解释器同anaconda在同一台电脑上并存,因为此时你的电脑里将同时有两个python运行环境,且这两个运
行环境没有通过环境管理器进行管理,其激活次序,即运行python脚本文件时正在使用的python解释器,或是尝试使用pip安装第三方库所指向的环境不易明确。
总之,作为课程的初学者,也是本手册后面的安排,我们将基于anaconda(conda+python解释器)进行介绍,包括环境配置,工具使用等,
所有代码也将会在这个环境下进行运行。
\subsection{Anaconda下载与安装}
Anaconda现已实现完善的公司化运营,可以访问其网站下载Anaconda:

https://www.anaconda.com/download

但通常,出于下载速度的考量,我们选择使用校园网联合镜像站的方式下载Anaconda,这不仅仅可以提供更快的安装包下载速度,同时其分发的
安装包也会将其中的默认镜像站替换成自己的分发站点,从而提供更高的pip与conda下载与更新速度。提供镜像站的组织有很多,如清华大学TUNA协会,中国科学技术大学所等高校及高校学生组织
所提供的下载源,此处以清华大学TUNA协会提供的下载源为例。

访问https://mirrors.tuna.tsinghua.edu.cn/anaconda/archive/ \footnote{此为双栈站点,可以自动选择IPv4或IPv6解析,将mirrors替换为mirrors6则只解析IPv6,替换为mirrors4则只解析IPv4,在校园网环境下,使用IPv6下载的文件不计流量。},
获取anaconda安装包列表。根据自己的操作系统版本和硬件架构,选择
下载对应的安装包体(如Windows-x86\_64对应AMD,
Intel处理器的Windows设备,MacOSX-arm64对应搭载Apple Silicon M1、M2、M3的设备)。
在当前的时间节点,建议选择Anaconda3-2023.09开头的安装包。如图\ref{fig:anacondaSelection}所示
\begin{figure}[htbp]
    \centering
    \includegraphics*[width=0.8\linewidth]{pic/tuna_anacondaDownload.png}
    \caption{Anaconda3安装包示例}
    \label{fig:anacondaSelection}
\end{figure}

对于不太熟悉命令行操作的同学来说,建议下载文件拓展名为pkg或exe的安装包,其可以提供图形化的安装界面和选项。具体安装过程在此不再赘述,
但请注意,\textbf{不要}把Anaconda安装在\dotuline{含有中文的目录}下,尤其注意你的Windows用户名是否带有中文。如果你不确定,可以直接将其安装在C盘外的其
他盘符\footnote{不要将Anaconda安装在C:/Program Files中},另外创建的文件夹中。

\subsection{环境变量与Spyder配置}

\subsubsection{环境变量配置}
配置环境变量的目的是令其他程序在运行python时能够直接找到base环境下的python,而不需要手动指定,同时也方便我们直接打开命令行就可以直接使用python解释器运行和调试程序。
对于windows和macOS而言,其配置环境变量的方式略有不同,下面分别介绍。

\textbf{Windows设备}

对于Windows设备而言,需要将anaconda安装目录下的bin/和Script

\subsubsection{Spyder}
安装流程结束后,在开始菜单(macOS:启动台)中寻找到Anaconda-Navigater,点击运行。
加载结束后,可以打开Spyder.
\end{document}